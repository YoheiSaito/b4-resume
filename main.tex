\documentclass[twocolumn, 9pt, a4j, dvipdfmx]{jsarticle}
%
\usepackage[top=20truemm,bottom=20truemm,left=20truemm,right=20truemm]{geometry}

%
\usepackage{url}
\usepackage{amsmath,amssymb}
\usepackage{type1cm}
\usepackage{color}
\usepackage{bm}
\usepackage{graphicx}
\usepackage{subfigure}
\usepackage{verbatim}
\usepackage{wrapfig}
\usepackage{ascmac}
\usepackage{makeidx}
\usepackage{enumerate}
\usepackage{algorithm}
\usepackage{algpseudocode}
\usepackage{multicol}
%
\usepackage{caption}
\captionsetup[figure]{format=plain, labelformat=simple, labelsep=period, font=bf}
\captionsetup[table]{format=plain, labelformat=simple, labelsep=period, font=bf}

%
\newcommand{\argmax}{\mathop{\rm arg~max}\limits}
\newcommand{\argmin}{\mathop{\rm arg~min}\limits}
\newcommand{\median}{\mathop{\rm \text{median}}\limits}
% \newcommand{\red}[1]{{#1}}
\newcommand{\red}[1]{\textcolor{red}{#1}}
% \newcommand{\diff}{\mathrm{d}}  %微分記号
% \newcommand{\divergence}{\mathrm{div}\,}  %ダイバージェンス
% \newcommand{\grad}{\mathrm{grad}\,}  %グラディエント
% \newcommand{\rot}{\mathrm{rot}\,}  %ローテーション
%

\pagestyle{empty}
\title{
    キーポイントパッチ抽出法を用いた\\
    三次元点群レジストレーションに関する諸検討
}
\author{16T2806E 齊藤 陽平}
\date{\today}

\begin{document}
\vspace{5cm}
\maketitle
\section{研究背景}
三次元点群はコンピュータビジョンの位置分野であり, 
物体を複数の離散的な点で表現したものである.
近年では三次元点群の取得できる安価なセンサの普及により
モデリングやロボディクス, 測量などの分野で三次元点群の
技術に対する需要が高くなっている. 
特に三次元点群処理は構成する点の数が莫大になり計算コストが高くなるため, 
高能率な手法が求められている.

実空間上の物体をレーザなどで測定し点群化するとき,
物体の影や裏側にはレーザが届かず一回で完全に物体の全面を一つの点群として
構成することは計測が不可能である.
そのため完全なモデルを作成するには複数の位置から取得した点群を一つにまとめる必要がある. 
このような場合に
複数の点群の相対関係を決定する問題をレジストレーションという. 
本稿ではレジストレーションにおいて重ねる基準となる点群をターゲット点群
$\mathcal{T}=\{\bm{q_1}, \bm{q_2}\ldots\bm{q_{N_t}}\}$
ターゲット点群に重なる点群をソース点群
$\mathcal{S}=\{\bm{p_1}, \bm{p_2}\ldots\bm{p_{N_s}}\}$とする. 

\section{レジストレーション手法と先行研究}
代表的なレジストレーションの手法にはICP, 特徴点のマッチング, 進化計算など
が挙げられる. 
中でも進化計算を使ったレジストレーション(ECR)では精度や成功確率が高い一方で
計算時間が長いというデメリットが存在する. 
進化計算レジストレーションでは式(\ref{FS})の評価関数を最小化する事で, 
最適な重なりとなる剛体変換$f$を求める.
ここで, 剛体変換$f$を構成するのに必要な要素は
$(\theta_x, \theta_y, \theta_z ,t_x, t_y, t_z)$の6次元であり, 
進化計算の遺伝子もこの6次元ベクトルである. 
\begin{equation}
    \label{FS}
    FS = \frac{1}{N_s} \sum_{\bm{p}_i \in \mathcal{S}}
        \| f(\bm{p}_i) - \argmin_{\bm{q}_j \in\mathcal{T}}(f(\bm{p}_i) - \bm{q}_j) \|^2
\end{equation}
計算時間が長い理由は評価関数で用いられる最近傍探索の回数が
$個体数 \times 世代数 \times N_s$と非常に多くなってしまうためである. 
そこで植西らはキーポイントパッチ(KPP)抽出法によるECR(KPP-ECR)を提案した.\cite{KPP}
この手法ではソース点群をキーポイントとその周辺の点に限定することで, 
ECRを100倍程度の高速化に成功した. 
しかし, 
KPP-ECRでは点群同士重なりのない場所にKPPを抽出すると
そのKPPがターゲット点群に惹きつけられてしまうため
そのKPPの位置に成功可能かどうかが深く関係する.
しかし, KPPが重なるかどうかは最適な剛体変換を求めるまで不明である. 
そこで重なるところのKPPだけを使う工夫が必要になる. 
しかし, 従来法ではKPPの位置に対しての最適な配置に関する考察は未だなされていない. 
本研究ではこのKPPの位置を自動的に補正する方法について初歩的な提案と
その実験・考察を記す. 
\section{提案手法}
本提案手法では予めキーポイントパッチを大量に抽出し, 
その中から使用するパッチを適宜選択することで , 
キーポイントパッチの位置を最適化の過程で変化させる.
各世代で評価関数が変動するため, 進化計算手法にはCMA-ESを用いる. 
次のようなステップで実行する. 
\begin{enumerate}[Step1.]
    \item キーポイント周囲の点をパッチ集合$\mathcal{A}$として抽出する
    \item CMA-ESのパラメータを設定する
    \item CMA-ESの平均を用いて, パッチを評価する 
    \item パッチの評価値からパッチ集団を更新する 
    \item パッチ集団の中から評価に使うパッチ$\mathcal{U}$を選択する 
    \item CMA-ESで生成した個体を$\mathcal{U}$を使って評価する
    \item CMA-ESのパラメータを更新する 
    \item 収束条件ならば終了, そうでなければStep3.に戻る
\end{enumerate}

Step3. パッチの評価では1世代に一度パッチ$\mathcal{A}_i$毎に
式(\ref{FS_PATCH})によって評価値を求める. 
ここで$cl$は$p_i$の最近傍の点番号である. 
$\bm{p}_{(i,n)}$は点$\bm{p}_i$の法線を示す. 
この評価関数では法線による値と距離による値の積で評価を行っている. 
理想的に重なる点群かどうかは形状の情報のみを利用することが望ましい. 
形状だけの情報で評価するとターゲット点群の穴の中のパッチが
稀に残ってしまうので今回は距離情報も加味した.
\begin{align}
    \label{FS_PATCH}
    \small
    FS_{p}(\mathcal{A}_i) = 
    \scriptsize
    \sqrt{1 -  
        \frac
        {\sum_{\bm{p}_i \in \mathcal{A}_i}|\bm{p}_{(i,n)} \cdot \bm{q}_{(cl,n)}|}
        {\text{count}(\mathcal{A}_i)} 
    } 
    \sqrt{
        \frac
        {\sum_{\bm{p_i}\in\mathcal{A}_i}\|\bm{p_i} - \bm{q_{cl}}\|^2}
        {\text{count}(\mathcal{A}_i)}
    }
\normalsize
\end{align}

Step4. パッチ集団の更新では世代$g$でのパッチ集団$\mathcal{A}^g$を
式(\ref{restrict},\ref{remove})で更新していく. 
更新によって重ならないパッチを選択肢から除外する. 
ここで$G$は終了世代数, $\bar{\mathcal{A}^g}, \sigma_{\mathcal{A}^g}$はそれぞれ
世代のパッチ評価値の平均, 分散である. 

\begin{align}
    \label{restrict}
    \mathcal{A}^{g+1}=
    \begin{cases}
        remove(\mathcal{A}^g) & N_u < \text{count}(remove(\mathcal{A}^g))\\
        \mathcal{A}^g & otherwise
    \end{cases}
\end{align}
\vspace{-2em}
\begin{align}
    \label{remove}
    remove(\mathcal{A}^g) = \{
        \mathcal{A}^g_i \in \mathcal{A}^g \mid 
        \bar{\mathcal{A}^g} + k\sigma_{\mathcal{A}^g} < FS_p(\mathcal{A}^g_i)
    \}
\end{align}
\vspace{-2em}
\begin{align}
    \label{trim-rate}
    k_g = \alpha + \beta \frac{G-g}{G+g}
\end{align}

Step5. パッチ選択は評価関数に使用するためのパッチを$N_u$個, 選択肢から選択する. 
評価が良いパッチも悪いパッチも選ぶために, 式(\ref{select},\ref{select-idx})
を利用して評価値全体でまんべんなく評価を選択する. 
ここで$\mathcal{A}_i^{(g+1) '}$は
評価値でソートされたパッチ集団$\mathcal{A}_i^{(g+1)}$であり, 
$L_g = \lfloor \frac{\text{count}(\mathcal{A}^{g+1})}{N_u} \rfloor$である. 
このように選択する事で, 一部のパッチのみが収束してしまうことを防ぐ. 
\begin{align}
    \label{select}
    \mathcal{U}^{g+1} = \bigcup_{i \in I^g} \mathcal{A}^{(g+1) '}_i 
\end{align}
\vspace{-2em}
\begin{align}
    \label{select-idx}
    I^g = \{L_g j + rand (0, L_g) | j \in \mathbb{Z} \cap [1, N_u]\}
\end{align}

\section{実験}
本手法と従来のKPP-ECRで実際にレジストレーションを行った. 
実験では時間短縮のため, 各点群に2mmのボクセルサンプリングを行った.  
実験は各データセットで30回づつ行い, 結果はその平均を表示する. 
実験パラメータとその結果を\ref{Dynamic-result}に示す. 
ここで結果の単位はメッシュ解像度であり, 成功はレジストレーション
結果と真の変換を行った点群との平均距離がターゲット点群のメッシュ解像度以下,
すなわちRMSEが1以下の時とした. 

\begin{table}[b]
    \footnotesize 
    \caption{提案手法と従来法のレジストレーション比較\label{Dynamic-result}}
    \begin{center}
    \begin{tabular}{l  c  c  c  c  c c c c c}\hline
実験 &  手法 & 成功率 &  RMSE & RMSE-s & RMSE-f \\ 
\hline \hline
    exp1 & 従来 & 0.87   & 2.66  & 0.87 & 14.34 \\
         & 提案 & 0.8    & 2.29  & 0.56 & 9.23 \\
    \hline
    exp2 & 従来 &   0    & 9.39  & -    & 9.39 \\
         & 提案 & 0.63   & 4.34  & 0.58 & 10.83\\
    \hline
    exp3 & 従来 &   0    & 2.05  & -    & 2.05 \\
         & 提案 & 0.93   & 1.59  & 0.25 & 20.31\\
    \hline
    exp4 & 従来 &   0    & 3.57  & -    & 3.57 \\
         & 提案 & 0.23   & 15.13 & 0.35 & 19.63\\
    \hline
    exp5 & 従来 &   1    & 0.418 & 0.42 & -    \\
         & 提案 &   1    & 0.312 & 0.31 & -    \\
    \hline
    exp6 & 従来 &   1    & 0.74  & 0.74  & -    \\
         & 提案 & 0.9    & 1.62  & 0.39  & 12.69\\
    \hline
    exp7 & 従来 &   1    & 0.17  & 0.17 &  -     \\
         & 提案 & 0.77   & 1.22  & 0.23 &  4.50 \\
    \hline
    exp8 & 従来 &   0    & 6.47  & -    &  6.47 \\
         & 提案 &   0    & 8.18  & -    &  8.18 \\
    \hline
    \end{tabular}
    \end{center} 
\end{table}      

\section{考察・結言}

\begin{thebibliography}{1}
\bibitem{KPP} 植西一馬, サンドバル ハイメ, 岩切宗利, 田中清 : 
"キーポイントパッチ抽出法を用いた高能率な進化計算による3次元点群レジストレーション", 
画像電子学会誌 The journal of the Institute of Image Electronics Engineers of Japan : 
visual computing, devices \& communications vol.47, no.2, pp.154-166, (2018)
\bibitem{stanford} "The Stanford 3D Scanning Repository." \url{http://
graphics.stanford.edu/data/3Dscanrep/} 
\bibitem{CMA-ES} Hansen, N. : 
"The CMA evolution strategy: a comparing review", Towards a new evolutionary 
computation. Advances on estimation of distribution algorithms, 
Springer, pp. 1769–1776, CiteSeerX 10.1.1.139.7369 (2006)
\end{thebibliography}
\clearpage
\end{document}
